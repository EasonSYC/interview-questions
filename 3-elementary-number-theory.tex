
\section{Elementary Number Theory}
\subsection{Primes and Divisibility}

\begin{problem}
    Define what is meant by a prime. Prove that there are infinitely primes. By considering \(4p_1 p_2 \cdots p_n + 3\), prove that there are infinitely primes of the form \(4n + 3\). Can we use the same proof for the primes of the form \(4n + 1\)?
\end{problem}

\begin{solution}
    For contradiction, let \(p_1, p_2, \ldots, p_n\) be a list of finitely many primes. Consider \(n = p_1 p_2 \cdots p_n + 1\) does not have any prime factors, contradicting with the fundamental theorem of arithmetic.

    A number of the form \(4n + 3\) must have a prime of the form \(4n + 3\) as its prime factor, and a similar argument follows.

    This does not work precisely in this form since a number of the form \(4n + 1\) does not necessarily have a prime factor of the form \(4n + 1\).
\end{solution}

\begin{problem}
    Is there a block of \(100\) integers, none of which are prime? What about precisely \(2\) of which are prime?
\end{problem}

\begin{solution}
    Consider \(101! + 2\), \(101! + 3\), \dots, \(101! + 101\). Note they are all not prime.

    Consider \(\brce{1, 2, \ldots, 100}\) has \(25\) primes.

    If \(\brce{a + 1, \ldots, a + 100}\) has \(n_a\) primes, then \(\brce{\para{a + 1} + 1, \ldots, \para{a + 100} + 1}\) has \(n_a\) or \(n_a \pm 1\) primes.

    This means from \(\brce{1, 2, \ldots, 100}\) having \(25\) primes to \(\brce{101! + 2, \ldots, 101! + 101}\) having \(0\) primes, it must have been through a block with only \(2\) primes.
\end{solution}

\begin{problem}
    Find all positive integers \(n\) for which \(n\) does not divide \(\para{n - 1}!\).
\end{problem}

\begin{solution}
    We can quickly convince ourselves that if \(n = ab\) for some \(a, b \in \NN\), \(1 < a, b < n\) and \(a \neq b\), then \(n\) divides \(\para{n - 1}!\).

    What remains is \(n = p\) or \(n = p^2\) for \(p\) a prime. The first one does not work. For the second one, it is possible for \(p > 2\) since \(1 \leq p < 2p \leq p^2 - 1\).
\end{solution}

\begin{problem}
    Let \(\totient\para{n}\) be the number of integers \(m\) where \(1 \leq m \leq n\) such that \(\gcd\para{m, n} = 1\). What is \(\totient\para{6}\)? What is \(\totient\para{p}\) where \(p\) is a prime? By considering the fractions
    \[
    \frac{1}{n}, \frac{2}{n}, \ldots, \frac{n}{n},
    \]
    prove that
    \[
        n = \sum_{d \divides n} \totient\para{d}. 
    \]
\end{problem}

\begin{solution}
    We have
    \[
        \totient\para{6} = \abs{\brce{1, 5}} = 2,
    \]
    and for primes \(p\),
    \[
        \totient\para{p} = p - 1.
    \]

    There are \(n\) fractions in this list, in this form.

    On the other hand, if we simply the fractions to the simple form (where the numerator and the denominators are coprime), then the denominator is a factor of \(n\), say \(d \divides n\). There are \(\totient\para{d}\) of such fractions since the numerator is coprime. This gives the identity as desired.
\end{solution}

\subsection{Modular Arithmetic}

\begin{problem}
    What is the last digit of \(3^{2022}\)? What is the remainder of \(2022^{2022}\) when divided by \(7\)?
\end{problem}

\begin{solution}
    Note the final digit of \(3^n\) is periodic:
    \[
        3, 9, 7, 1, 3, 9, 7, 1, \ldots
    \]
    and since \(2022 \equiv 2 \pmod{4}\), so the final digit is \(9\).

    Alternatively, the period \(4\) can be seen from \(\gcd\para{3, 10} = 1\), and that \(\totient\para{10} = 4\), by Euler-Fermat.

    Note that \(2022 \equiv -1 \pmod{7}\), so
    \[
        2022^{2022} \equiv \para{-1}^{2022} \equiv 1 \pmod{7}.
    \]
\end{solution}

\begin{problem}
    Are there any squares of the form \(4n + 2\)? What about \(4n + 3\)? Consider the equation \(a^2 + b^2 = c^2\) where \(a, b, c\) are positive integers. What can be said about the parity of \(a, b\) and \(c\)? What about Pythagorean quadruples or quintuples?
\end{problem}

\begin{solution}
    No, since quadratic residues modulo \(3\) are \(0\) and \(1\).

    \(\para{a, b, c} \equiv \para{0, 0, 0}, \para{0, 1, 1}, \para{1, 0, 1} \pmod{2}\) but not \(\para{1, 1, 0}\) by modulo \(4\).

    For \(a, b, c, d\), it is possible to have \(a, b, c\) all even, one odd, but not two odds or three odds by modulo \(4\).

    For \(a, b, c, d, e\), it is possible to have \(a, b, c\) all even, one odd, or all odds.
\end{solution}

\begin{problem}
    Is \(123454321\) a multiple of \(9\)? What about \(12321\)? Show that a positive integer \(n\) is a multiple of \(9\) if and only if the sum of its digits is a multiple of \(0\). The number \(2^{29}\) is a nine-digit number with all digits distinct. Which digit is missing?
\end{problem}

\begin{solution}
    No for \(123454321\), since \(1 + 2 + 3 + 4 + 5 + 4 + 3 + 2 + 1 = 25\) is not a multiple of \(9\).
    
    Yes for \(12321\), since \(1 + 2 + 3 + 2 + 1 = 9\) is a multiple of \(9\).

    A number \(m = \overline{x_n x_{n - 1} \cdots x_1 x_0}\) satisfies
    \begin{align*}
        m &\equiv \sum_{k = 0}^{n} 10^{k} x_k \\
        &\equiv \sum_{k = 0}^{n} 1^{k} x_k \\
        &\equiv \sum_{k = 0}^{n} x_k \pmod{9}
    \end{align*}
    and this proves the statement.

    We notice that
    \begin{align*}
        2^{29} &\equiv \para{2^3}^9 \cdot 2^2 \\
        &\equiv \para{-1}^{9} \cdot 4 \\
        &\equiv -4 \pmod{9} 
    \end{align*}
    and hence the digit missing is \(4\).
\end{solution}

\subsection{Number Base}

\begin{problem}
    What is \(100_{10}\) in base 3? What is \(10201_{11}\) in base 3? What is \(\para{\frac{1}{2}}_{10}\) in base 3? Describe the Cantor set in base 3.
\end{problem}

\begin{definition}[Cantor Set]
    The Cantor set is defined as (the limit of) the following fractal, where each line segment is an open interval, and the first represents the closed interval \(\para{0, 1}\).

    \begin{center}
        \begin{tikzpicture}[scale=2]
            \foreach \order in {0,...,8}
                \draw[yshift=-\order*5pt]  l-system[l-system={cantor set, axiom=F, order=\order, step=100pt/(3^\order)}];
        \end{tikzpicture}
    \end{center}

    Every time the middle \(\frac{1}{3}\) of a line segment is removed.
\end{definition}

\begin{solution}
    \(100_{10} = \para{10_{10}}^2 = \para{101_{3}}^2 = 10201_{3}\).

    Alternatively, \(100 = 81 + 2 \times 9 + 1 = 3^4 + 2 \times 3^2 + 1\) so \(100_{10} = 10201_{3}\).

    \(121_{11} = \para{11_{11}}^2 = \para{12_{10}}^2 = \para{110_{3}}^2 = 12100_3\).

    We notice that
    \begin{align*}
        \frac{1}{2} &= \frac{1}{3} \cdot \frac{1}{1 - \frac{1}{3}} \\
        &= \frac{1}{3} + \frac{1}{3^2} + \frac{1}{3^3} + \cdots
    \end{align*}
    hence
    \[
        \para{\frac{1}{2}}_{10} = 0.111\ldots_{3}.
    \]

    The Cantor set contains precisely the numbers in \([0, 1]\) without any \(1\)1 in their ternary expansion. If we define the line segments to be closed intervals, then it contains precisely the numbers in \([0, 1]\) with finitely many \(1\)s in their ternary expansion.
\end{solution}